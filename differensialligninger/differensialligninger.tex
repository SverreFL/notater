\chapter{Differensialligninger}
Differensialligninger er veldig nyttig for å modellere virkeligheten. Det er ofte enklere å beskrive hva som påvirker endringsraten i en variabel enn dens absolutte verdi. Vi kan for eksempel anta at vekstraten til en populasjon er proposjonal med størrelsen på populasjonen (eller med et mål på ledig kapasistet). Differensialligninger er ligninger der den ukjente er en funksjon og der det inngår en derivert av funksjonen i ligningen. Anvendt matematikk er i praksis modellering og differensialligninger er måten modellene er konstruert.
\section{Modeller}
Modeller er forenklede representasjoner av virkeligheten og benyttes i veldig mange sammenhenger. For eksempel gir kart en nedskalert og to-dimensjonal representasjon av hvordan verden ser ut. Hvordan kartet er utformet avhenger av hva det skal brukes til. For et turkart er gjerne terreng og høydeforskjell noen vesentlige aspekter ved virkeligheten som fremheves, mens det for t-bane-kart er mer vesentlig å representere relativ plassering til stasjoner og hvilken bane som går hvor. Det illustrerer at det er uvesentlig om en modell er sann eller ikke siden det per konstruksjon er en fordreining av virkeligheten; det vesentlige er om det er et nyttig verktøy for det formålet det benyttes til. 
\subsection{Formålet med modeller}
Modeller kan ha ulike formål. Vi kan betrakte det som rene tankeeksperiment der vi belyser spesifikke mekanismer ved å holde andre aspekter konstant, analogt til fysiske laboratoriumsforsøk. Dette kan gi en mer systematisk måte å tenke på og gi økt forståelse ved å se på de enkelte delene i isolasjon før de settes sammen til kaoset som utgjør virkeligheten.

Andre modeller brukes til å svare på helt konkrete spørsmål. Da vil vi fokusere på de aspektene ved virkeligheten som påvirker løsningen. For en passasjer på t-banen som forsøker å navigere er for eksempel de eksakte geografiske avstandene og høydeforskjellene lite vesentlig. Ved å se bort fra dette kan vi gi en enklere representasjon som er mer egnet for å navigere. Siden modellene er et verktøy må de være anvendelig for formålet; de bør være så enkle som mulig, men ikke enklere. Det er også en fordel at modellen er enkel slik at brukeren kan vurdere troverdigheten til løsningen.
\subsection{Matematiske modeller}
Det finnes som nevnt mange typer modeller som brukes i ulike sammenhenger, men jeg vil fokusere på matematiske modeller som i praksis kan representeres med et ligningssystem som beskriver relasjonen mellom størrelsene i modellen. Modellene er internt logiske konsistente slik at de gir entydige svar på spørsmål vi stiller, uavhengig av hvem som bruker modellen. Hvordan modellen er konstruert og i hvilken grad de svarene vi får ut fra modellen kan hjelpe oss å svare på våre egentlige spørsmål om virkeligheten krever derimot subjektiv vurdering, og forutsetter både kunnskap og forståelse av modellen samt den virkeligheten som den skal belyse. Modellen er som nevnt bare et verktøy og har ingen egen vilje. I likhet med en hammer kan den både være nyttig og gjøre stor skade. 

Si noe om variabel og parametre. Si noe om usikkerhet + data/statistikk.

Si noe om hvordan vi håndetere tid. System utvilkler seg over tid. Ulike tilstand på ulike tidspunkt. Likevekt, diskret, kontinuerlig. 
\section{Differensligninger}
Differensligninger er modeller i diskret tid der variabelen $x$ i en periode $t$ avhenger av verdien den tok i perioden før.\footnote{Tror det også kan avhenge av tidligere periode. Avgrenser til å se på første ordens differensligninger}. Generelt kan ligningen skrives på formen
\begin{align}
F(t,x_t, x_{t-1})=0.
\end{align}
Dersom denne ligningen kan løses eksplisitt for $x_t$ får vi funksjonen
\begin{align}
f(x_t)=f(t,x_{t-1}).
\end{align}
Hvis vi kjenner $f(\cdot)$ og en initialverdi $x_0$ kan vi finne verdi av variabelen på hver tidspunkt gjennom sekvensiell substitusjon,
\begin{align}
&f(x_1) = f(1,x_0) \\
&f(x_2) = f(2, x_1) = f(2,(f(1,x_0)) \\
&f(x_3) = f(3,x_2) = f(3,f(2,(f(1,x_0)))) \\
&\cdots
\end{align}
Dette er praktisk for simulering, men for å studere løsningen vil vi gjerne ha et eksplisitt utrykk for $f(x_t)$ som funksjon av $t$ og noen parametre. Vi vil studere om variabelen konvergerer mot likevekt, hvorvidt likevekten er stabil og hvordan dette avhenger av parametre. Kanskje mer om dette en annen gang.
\section{Første ordens differensialligninger}
Jeg vil klassifisere ulike typer differensialligninger og finne (og memorisere...) algoritmer for å løse de analytisk. Litt terminologi:
\begin{itemize}
\item Ordinær hvis den kun har én uavhengig variabel (i praksis tid $t$) , ellers partiell
\item Orden er høyeste grad av deriverte som inngår i ligningen
\item Lineær hvis avhenger lineært av argumentene (de deriverte), $a_1 y' + \dots a_n y^{(n)}=y$. Alltid enklere å jobbe med lineære ligninger..
\item Ligningssystem hvis flere ukjente, f.eks. både $y(t)$ og $x(t)$. 
\item En løsning er en funksjon som tilfredstiller lingingen på et intervall. Kan representeres som en kurve i $ty$-diagram. Vanskelig å finne løsninger, men kan alltids sjekke om det utgjør en løsning ved å substituere det inn i differensialligningen.
\item I praksis er det en familie av løsninger og finner unik løsning gjennom kunnskap om initialverdi. Vet at den er unik hvis kurvene ikke krysser... 
\end{itemize}
\subsection{Generelt om løsninger}
Retningsdiagram, grafer, generelle løsninger, spesifikk løsning dersom informasjon at krysser gjennom gitt punkt i $tx$-diagram (i praksis initialverdi-problem) med mer.

Generell form på første ordens differensialligninger
\begin{align}
\dot{x}=F(t,x)
\end{align}

Løsning på et intervall $I$ dersom for alle $t \in I$ så holder ligningen over når vi deriverer $x(t)$... Kan ha closed form eksplisitt løsning, men dette er bare i spesialtilfelle. Kan likevel undersøke egenskaper til løsningen og hvordan det avhenger av parametre, samt finne verdier numerisk og visualisere egenskaper ved systemet.
\subsection{Separable}
Vi kan begynne med å betrakte separable første ordens differensialligninger som generelt kan utrykkes på formen
\begin{align}
\dot{x}=f(t)g(x).
\end{align}
Disse kan løses gjennom \footnote{bør kunne vise dette mer formelt med substitusjon i integrasjon. Kan huske ved at $\dot{x}=\frac{dx}{dt}$ og gjøre vanlig algebra, men ukomfortabel med dette}
\begin{align}
\int\frac{1}{g(x)}dx &= \int f(t)dt + C \\
G(x) &= F(t) + C \\
x &= G^{-1}(F(t)+C)
\end{align}
Merk også at det er konstante løsninger $x(t)=a$ for alle røtter av $g(x)$, altså verdier $x=a$ der $g(a)=0$. Enkelt eksempel:
\begin{align}
\dot{x}&=f(t)x, \quad x>0 \\
\int \frac{1}{x}dx &= \int f(t)dt + C_1 \\
\ln x &= F(t)+C_1 \\
x &= e^{F(t)}e^{C_1} \\
x &= Ce^{F(t)} 
\end{align}
\subsection{Lineære}
Vi kan også finne eksplisitt løsning for lineære første ordens differensialligninger på formen
\begin{align}
\dot{x}+a(t)x=b(t).
\end{align}
For å løse disse bruker vi et triks der vi skalerer begge sider med en såkalt integrerende konstant slik at venstresiden får formen $\frac{d}{dt}[u(t)v(t)] = v\dot{u}+\dot{v}u$. Vi ganger altså med $v$ der $\frac{d}{dt}va(t)=v$.  Eksempel der $a(t)=a$,
\begin{align}
\dot{x}+ax&=b(t) \\
e^{at}[\dot{x}+ax]&=e^{at}b(t) \\
e^{at}\dot{x}+ae^{at}x &=e^{at}b(t) \\
\frac{d}{dt}[e^{at}x] &= e^{at}b(t) \\
e^{at}x &= \int e^{at}b(t)dt + C
\end{align}